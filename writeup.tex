%%%%%%%%%%%%%%%%%%%%%%%%%%%%%%%%%%%%%%%%%%%%%%%
%%%This is a science homework template. Modify the preamble to suit your needs. 
%The junk text is   there for you to immediately see how the headers/footers look at first 
%typesetting.


\documentclass[12pt]{article}

%AMS-TeX packages
\usepackage{amssymb,amsmath,amsthm} 
%geometry (sets margin) and other useful packages
\usepackage[margin=1.25in]{geometry}
\usepackage{graphicx,placeins}


%
%Redefining sections as problems
%
\makeatletter
\newenvironment{problem}{\@startsection
       {section}
       {1}
       {-.2em}
       {-3.5ex plus -1ex minus -.2ex}
       {2.3ex plus .2ex}
       {\pagebreak[3]%forces pagebreak when space is small; use \eject for better results
       \large\bf\noindent{Problem }
       }
       }
       {%\vspace{1ex}\begin{center} \rule{0.3\linewidth}{.3pt}\end{center}}
       \begin{center}\large\bf \ldots\ldots\ldots\end{center}}
\makeatother


%
%Fancy-header package to modify header/page numbering 
%
\usepackage{fancyhdr}
\pagestyle{fancy}
%\addtolength{\headwidth}{\marginparsep} %these change header-rule width
%\addtolength{\headwidth}{\marginparwidth}
\lhead{Problem \thesection}
\chead{} 
\rhead{\thepage} 
\lfoot{\small\scshape Machine Learning in Complex Domains} 
\cfoot{} 
\rfoot{\footnotesize PS \#3} 
\renewcommand{\headrulewidth}{.3pt} 
\renewcommand{\footrulewidth}{.3pt}
\setlength\voffset{-0.25in}
\setlength\textheight{648pt}

%%%%%%%%%%%%%%%%%%%%%%%%%%%%%%%%%%%%%%%%%%%%%%%

%
%Contents of problem set
%    
\begin{document}

\title{MLCD 3: Posterior Inference}
\author{Mike Smith and Elan Hourticolon-Retzler}

\maketitle

\thispagestyle{empty}

%%%%%%%%%%%%%%%%%%%%%%%%%%%%%%%%%%%%%%%%%%%%%%%%%
\section*{2 Variational Inference}
\noindent {\bf(a) Changing Day 1 Ice creams from 2 to 1...}\\

%%%%%%%%%%%%%%%%%%%%%%%%%%%%%%%%%%%%%%%%%%%%%%%%%
\section*{3 Analyzing Multiple Text Corpora}
\noindent {\bf(a) Changing Day 1 Ice creams from 2 to 1...}\\

%%%%%%%%%%%%%%%%%%%%%%%%%%%%%%%%%%%%%%%%%%%%%%%%%
\section*{4 Collapsed Gibbs Sampler Implementation}
\noindent {\bf(a) Changing Day 1 Ice creams from 2 to 1...}\\

%%%%%%%%%%%%%%%%%%%%%%%%%%%%%%%%%%%%%%%%%%%%%%%%%
\section*{5 Blocked Gibbs Sampler}
\noindent {\bf(5.1) Analysis Questions}\\
Given the full expanded form of the likelihood:\\
\begin{align}
P (z, x, c, w\mid \alpha, \beta, \lambda) &= P (w \mid z, x, c, \beta)P (z\mid \alpha)P(x\mid \lambda) \nonumber\\
&=\prod_{k}(\frac{\Gamma(\sum_{w} \beta)}{\Gamma(\sum_{w} n^{k}_{w}+\beta)} \prod_{w} \frac{\Gamma(n^{k}_{w}+\beta))}{\Gamma(\beta)} )\nonumber\\
&\times \prod_{c}\prod_{k}(\frac{\Gamma(\sum_{w} \beta)}{\Gamma(\sum_{w} n^{c,k}_{w}+\beta))} \prod_{w} \frac{\Gamma(n^{c,k}_{w}+\beta))}{\Gamma(\beta)} )\nonumber\\
&\times \prod_{d}(\frac{\Gamma(\sum_{k} \alpha)}{\Gamma(\sum_{k} n^{d}_{k}+\alpha)} \prod_{k} \frac{\Gamma(n^{d}_{k}+\alpha))}{\Gamma(\alpha)} ) \times \prod_{(d,i)} P(x_{d,i} \mid \lambda)\nonumber\\
\end{align}

\begin{align}
P (z_{d,i}, x_{d,i} \mid z-z_{d,i},x-x_{d,i},c,w;\alpha, \beta, \lambda) &= \frac{P (z, x, c, w\mid \alpha, \beta, \lambda)}{P (z-z_{d,i}, x-x_{d,i}, c, w\mid \alpha, \beta, \lambda)} \nonumber\\
\end{align}

\begin{align}
P &(z_{d,i}=k, x_{d,i} = 0 \mid z-z_{d,i},x-x_{d,i},c,w;\alpha, \beta, \lambda)  \nonumber\\
&=\frac{\frac{\Gamma(n^{k}_{w_{d,i}} +1+\beta)}{\Gamma(1+\sum_{w} n^{k}_{w}+\beta)}  \frac{\Gamma(n^{d}_{k} +1+\alpha)}{\Gamma(1+\sum_{k'} n^{k'}_{w}+\alpha)}}{\frac{\Gamma(n^{k}_{w_{d,i}} +\beta)}{\Gamma(\sum_{w} n^{k}_{w}+\beta)}  \frac{\Gamma(n^{d}_{k} +\alpha)}{\Gamma(\sum_{k'} n^{k'}_{w}+\alpha)}} P(x_{d,i} = 0 \mid \lambda)\nonumber\\
&= \frac{n^{k}_{w_{d,i}}+\beta}{n^{k}_{*}+V\beta}  \frac{n^{d}_{k} +\alpha}{n^{d}_{*} +K\alpha} (1-\lambda) 
\end{align}

\begin{align}
P &(z_{d,i}=k, x_{d,i} = 1 \mid z-z_{d,i},x-x_{d,i},c,w;\alpha, \beta, \lambda)  \nonumber\\
&=\frac{\frac{\Gamma(n^{c,k}_{w_{d,i}} +1+\beta)}{\Gamma(1+\sum_{w} n^{c,k}_{w}+\beta)}  \frac{\Gamma(n^{d}_{k} +1+\alpha)}{\Gamma(1+\sum_{k'} n^{k'}_{w}+\alpha)}}{\frac{\Gamma(n^{c,k}_{w_{d,i}} +\beta)}{\Gamma(\sum_{w} n^{c,k}_{w}+\beta)}  \frac{\Gamma(n^{d}_{k} +\alpha)}{\Gamma(\sum_{k'} n^{k'}_{w}+\alpha)}} P(x_{d,i} = 1 \mid \lambda)\nonumber\\
&= \frac{n^{c,k}_{w_{d,i}}+\beta}{n^{c,k}_{*}+V\beta}  \frac{n^{d}_{k} +\alpha}{n^{d}_{*} +K\alpha} (\lambda) 
\end{align}


%%%%%%%%%%%%%%%%%%%%%%%%%%%%%%%%%%%%%%%%%%%%%%%%%
\section*{6 Text Analysis with MCLDA}
\noindent {\bf(1)} For each chain the training likelihood increases (monotonically, save for minor fluctuations)  till it plateaus around 300 iterations. Similarly the likelihood for test data decreases till it plateaus however with much more variation. \\
\\
\noindent {\bf(4)} TOPICS \\
\\
\noindent {\bf(5)} Lambdas\\
\\
\noindent {\bf(6)} Words \\
\noindent {\bf(a)} topics \\
\\
\noindent {\bf(b)} Varying $\lambda$ controls the amount we sample from global vs corpus dependent distribution of words in topics. Test performance increases with $\lambda$ initially but then peaks around .75 before decreasing again. This implies that there is an optimal lambda that is between always sampling from the global and always sampling from collection dependent. This intuitively makes sense because with $\lambda = 0$ this would imply that each bag of words for a topic is uniform between collections. The other extreme with $\lambda=1$ would imply that there was zero benefit to generalizing topics across collections and that the training data contained every word per collection per topic that it would see in test data. Everywhere in between implies that there are differences in which words each collection uses to describe a topic but there is still benefit to considering the generalized corpus.

Unsurprisingly, the likelihood of training data is maximized by setting $\lambda$ to 1.0 since there is no need to generalize with training data.
 \\
\\
\noindent {\bf(a)} alpha/beta \\
\\
%%%%%%%%%%%%%%%%%%%%%%%%%%%%%%%%%%%%%%%%%%%%%%%%%
\section*{7 Variational Inference}
\noindent {\bf(7.1) Derivation}\\
We wish to derive the update equations for variational inference, as computing $p(\theta, c, \phi, z, x \mid w, \alpha, \beta, \lambda)$ for exact inference is intractable.  By definition of a conditional probability we have:
\begin{equation*}
\frac{p(\theta, c, \phi, z, x, w, \mid \alpha, \beta, \lambda)}{p(w \mid \alpha, \beta, \lambda)}
\end{equation*}
From the graphical model of our distribution at the top of page 6 of the homework PDF, it is easy to mutilate the graph of P to arrive at a more tractable family of approximate Q distributions.  We omit the $w$ node, and omit the edges between $\theta$, $z$, $\phi$, and $x$.  We also introduce free variational parameters $\gamma$, $\delta$, $\zeta$, and $\eta$ respectively on our factors of Q, themselves distributions.  $q(\theta \mid \gamma) \sim \mathrm{Dirichlet}(\gamma)$, $q(z \mid \delta) \sim \mathrm{Multinomial}(\delta)$, $q(\phi \mid \zeta) \sim \mathrm{Dirichlet}(\zeta)$, and $q(x \mid \eta) \sim \mathrm{Bernoulli}(\eta)$.\\
We note that we want to minimize the KL Divergence from $q$ to $p$, which is
\begin{equation*}
\mathrm{KL}(q\|p) = \ln(Z) - \mathrm{E}_q (\log p) - \mathrm{E}_q (\log q) 
\end{equation*}
and this is equivalent to maximizing the energy functional:
\begin{equation*}
F = \mathrm{E}_q (\log p) + \mathrm{E}_q (\log q)
\end{equation*}
We know that our joint of $p$ factorizes as the following:
\begin{equation*}
p(\theta, c, \phi, z, x, w, \mid \alpha, \beta, \lambda) = p(z \mid \theta) p(\theta \mid \alpha) p(x \mid \lambda) p(\phi \mid \beta) p(w \mid z, x, c, \phi)
\end{equation*}
and that our denominator in the conditional factorizes as the following:
\begin{equation*}
p(w, \mid \alpha, \beta, \lambda) = p(w, \mid \alpha, \beta, \lambda)
\end{equation*}
and that $p(w \mid z, x, c, \phi)$ factorizes as the following:
\begin{equation*}
p(w \mid z, x, c, \phi) = \prod_{(d,i)} ( p(w_{d,i} \mid \phi_k)^{I(x_{d,i} = 0)} p(w_{d,i} \mid \phi_{k}^{c})^{I(x_{d,i} = 1)} )
\end{equation*}
and that $q$ factorizes\ as the following:
\begin{equation*}
q(\theta, z, \phi, x \mid \gamma, \delta, \zeta, \eta) = q(\theta \mid \gamma) \prod_d q(z \mid \delta) q(\phi \mid \zeta) q(x \mid \eta)
\end{equation*}
so we can plug these into our definition of the energy functional.
\begin{align*}
F &= \mathrm{E}_q \{\log p(z \mid \theta) + \log p(\theta \mid \alpha) + \log p(x \mid \lambda) + \log p(\phi \mid \beta) \\
&+ \log p(w \mid z, x, c, \phi) - \log p(w, \mid \alpha, \beta, \lambda) \} \\
&+ \mathrm{E}_q \{ \log q(\theta \mid \gamma) + \sum_d \left( \log q(z \mid \delta) + \log q(\phi \mid \zeta) + \log q(x \mid \eta) \right) \} 
\end{align*}
Now, we can calculate the individual terms by taking the logarithm of each, also moving the expectation to each term, as the expectation of a sum is the sum of expectations.\\
\noindent \textit{Note that some of these equations were taken from\\ http://machinelearning.wustl.edu/mlpapers/paper\_files/BleiNJ03.pdf }
\begin{align*}
\mathrm{E}_q \log p(z \mid \theta) &= \sum\limits_{i=1}^{N_d} \sum\limits_{k=1}^K (\alpha - 1)(\Psi(\gamma_k) - \Psi(\sum_{j=1}^k \gamma_j) ) \\
\mathrm{E}_q\log p(\theta \mid \alpha) &= \log \Gamma (K \alpha) - \sum\limits_{k=1}^K \log \Gamma(\alpha) + \sum\limits_{k=1}^K (\alpha - 1)(\Psi(\gamma_k) - \Psi(\sum_{j=1}^K \gamma_j) )\\
\mathrm{E}_q\log p(x \mid \lambda) &= ??\\
\mathrm{E}_q\log p(\phi \mid \beta) &= \log \Gamma (K \beta) - \sum\limits_{k=1}^V \log \Gamma(\beta) + \sum\limits_{k=1}^V (\beta - 1)(\Psi(\zeta_k) - \Psi(\sum_{j=1}^V \zeta_j) )\\
\mathrm{E}_q\log p(w \mid z, x, c, \phi) &= \sum_{d,i} \left( I(x_{d,i} = 0) p(w_{d,i} \mid \phi_k) + I(x_{d,i} = 1) p(w_{d,i} \mid \phi_{k}^{c})  \right)\\
\mathrm{E}_q\log p(w, \mid \alpha, \beta, \lambda) &= ??\\
\mathrm{E}_q\log q(\theta \mid \gamma) &= \log \Gamma (\sum\limits_{k=1}^K \gamma_k) - \sum\limits_{k=1}^K \log \Gamma(\gamma_k) + \sum\limits_{k=1}^K (\gamma_k - 1)(\Psi(\gamma_k) - \Psi(\sum_{j=1}^K \gamma_j) )\\
\mathrm{E}_q\log q(z \mid \delta) &= \sum\limits_{i=1}^{N_d} \sum\limits_{k=1}^K \delta_{i,k} \log \delta_{i,k} \\
\mathrm{E}_q\log q(\phi \mid \zeta) &= \log \Gamma (\sum\limits_{k=1}^V \zeta_k) - \sum\limits_{k=1}^V \log \Gamma(\zeta_k) + \sum\limits_{k=1}^V (\zeta_k - 1)(\Psi(\zeta_k) - \Psi(\sum_{j=1}^V \zeta_j) )\\
\mathrm{E}_q\log q(x \mid \eta) &= ??\\
\end{align*}
%%%%%%%%%%%%%%%%%%%%%%%%%%%%%%%%%%%%%%%%%%%%%%%%%

%%%%%%%%%%%%%%%%%%%%%%%%%%%%%%%%%%%%%%%%%%%%%%%%%
\end{document}
